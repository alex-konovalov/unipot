%%%%%%%%%%%%%%%%%%%%%%%%%%%%%%%%%%%%%%%%%%%%%%%%%%%%%%%%%%%%%%%%%%%%%%%%%
%%
%W  preface.tex            UNIPOT documentation             Sergei Haller
%%
%H  $Id: preface.tex,v 2.5 2004/10/21 21:36:20 gc1007 Exp $
%%
%Y  Copyright (C) 2000-2002, Sergei Haller
%Y  Arbeitsgruppe Algebra, Justus-Liebig-Universitaet Giessen
%%
%%

%%%%%%%%%%%%%%%%%%%%%%%%%%%%%%%%%%%%%%%%%%%%%%%%%%%%%%%%%%%%%%%%%%%%%%%%%
\Chapter{Preface}

\indextt{Unipot}
{\Unipot} is a package for {\GAP4} \cite{GAP4}. The  version 1.0
of   this  package  was   the  content  of  my  diploma   thesis
\cite{SH2000}.

Let $U$ be a unipotent  subgroup of  a  Chevalley  group of Type
$L(K)$.  Then it is generated by  the  elements $x_r(t)$ for all
$r\in  \Phi^+,t\in K$. The roots of the underlying  root  system
$\Phi$  are  ordered  according  to  the height  function.  Each
element of  $U$ is  a product of the  root elements $x_r(t)$. By
Theorem 5.3.3 from  \cite{Carter72}  each element of $U$ can  be
uniquely  written  as a product  of root  elements with roots in
increasing order. This unique form is called the canonical form.


The main  purpose  of this  package is to compute the  canonical
form of an element of the group $U$. For we have implemented the
unipotent subgroups  of Chevalley groups and  their elements  as
{\GAP} objects and  installed  some  operations  for  them.  One
method for the operation `Comm'  uses the Chevalley's commutator
formula, which we have implemented, too.


%%%%%%%%%%%%%%%%%%%%%%%%%%%%%%%%%%%%%%%%%%%%%%%%%%%%%%%%%%%%%%%%%%%%
\Section{Root Systems}

We  are  using  the  root  systems and  the  structure constants
available in {\GAP} from  the simple  Lie algebras. We  also are
using the same ordering of roots available in {\GAP}.

Note that the structure constants in {\GAP4.1} are not generated
corresponding  to  a Chevalley  basis, so  computations  in  the
groups of  type $B_l$  may produce an  error and computations in
groups  of  types  $B_l$,  $C_l$  and $F_4$ may  lead  to  wrong
results. In the groups of other types  we haven't seen any wrong
results but can not guarantee that all results are correct.

Since the revision  4.2  of  {\GAP} the  structure constants are
generated corresponding to a  Chevalley basis, so that they meet
all our assumptions.

Therefore  the package  requires at least  the  revision  4.2 of
{\GAP}.

Beginning with version 1.2 of \Unipot, the new package loading
mechanism of \GAP4.4 is used and therefore, \GAP4.4 is required.


%%%%%%%%%%%%%%%%%%%%%%%%%%%%%%%%%%%%%%%%%%%%%%%%%%%%%%%%%%%%%%%%%%%%
% \Section{Future of Unipot}
% 
% In one of the future versions of the package {\Unipot} we plan to
% implement some other features. Here is a small list of them:
% \beginlist%unordered
% \item{--} {\GAP4.2} provides special root system objects. We should use
%           them.
% \item{--} Provide some root systems in common notations (like Carter or
%           Bourbaki).
% \item{--} Allow the user to provide his own table of structure constants.
% \item{--} Provide whole Chevalley groups as {\GAP} objects
% \item{--} Provide root subgroups
% \item{--} The elements of Chevalley groups should act on the underlying
%           simple Lie algebra as automorphisms
% \item{--} There are many known properties of the Chevalley groups and
%           their unipotent subgroups like simplicity, central series, etc.
%           Implement them.
% \endlist                                                                                                      


%%%%%%%%%%%%%%%%%%%%%%%%%%%%%%%%%%%%%%%%%%%%%%%%%%%%%%%%%%%%%%%%%%%%%%%%%
\Section{Citing Unipot}

If you  use {\Unipot} to solve a problem or publish some  result
that was partly obtained  using {\Unipot}, I would appreciate it
if you would  cite {\Unipot},  just  as you  would  cite another
paper that you used. (Below is a sample citation.) Again I would
appreciate if you could inform me about such a paper.

Specifically, please refer to:
 
\begintt
[Hal02] Sergei Haller. Unipot --- a system for computing with elements
        of unipotent subgroups of Chevalley groups, Version 1.2.
        Justus-Liebig-Universitaet Giessen, Germany, July 2002. 
        (http://...)
\endtt
 
(Should the reference style require full addresses please use:
``Arbeitsgruppe Algebra,
  Mathematisches Institut,
  Justus-Liebig-Universit{\accent127a}t Gie{\ss}en,
  Arndtstr. 2,
  35392 Gie{\ss}en, Germany'')



%%%%%%%%%%%%%%%%%%%%%%%%%%%%%%%%%%%%%%%%%%%%%%%%%%%%%%%%%%%%%%%%%%%%%%%%%
%%
%E  preface.tex . . . . . . . . . . . . . . . . . . . . . . . . ends here



%%%%%%%%%%%%%%%%%%%%%%%%%%%%%%%%%%%%%%%%%%%%%%%%%%%%%%%%%%%%%%%%%%%%%%%%%
%%
%W  install.tex            UNIPOT documentation             Sergei Haller
%%
%H  $Id: install.tex,v 2.1 2000/07/13 11:38:36 gc1007 Exp $
%%
%%

%%%%%%%%%%%%%%%%%%%%%%%%%%%%%%%%%%%%%%%%%%%%%%%%%%%%%%%%%%%%%%%%%%%%%%%%%
\Chapter{Installing and Loading unipot}

This appendix describes the procedure of installing the share package

Installing `unipot' should be easy once you have installed {\GAP} itself.
We assume here that you want to install `unipot' in its standard
location, which is in the ``pkg'' subdirectory of the main {\GAP4}
installation.


%%%%%%%%%%%%%%%%%%%%%%%%%%%%%%%%%%%%%%%%%%%%%%%%%%%%%%%%%%%%%%%%%%%%%%%%%
\Section{Overview}

You have to perform the following steps to install `unipot':

\beginlist
\item{--} Get the sources.
\item{--} Unpack the sources with the <unzoo> utility.
\item{--} Optionally edit the ALLPKG file so that the `unipot'
     documentation will be available when {\GAP} starts up.
\endlist                                                                                                      


%%%%%%%%%%%%%%%%%%%%%%%%%%%%%%%%%%%%%%%%%%%%%%%%%%%%%%%%%%%%%%%%%%%%%%%%%
\Section{What you need to install unipot}

Being a share package for {\GAP4} and implemented in the {\GAP4}
language, `unipot' of course needs at least {\GAP} version 4.2.
(See Preface.1 why you shouldn't use it with {\GAP4.1}.)
`Unipot' runs on any system supporting {\GAP4}. It is tested with {\GAP4.2},
but should work with any non-beta version of {\GAP4} (with exceptions
stated in Preface.1 for {\GAP4.1}).


%%%%%%%%%%%%%%%%%%%%%%%%%%%%%%%%%%%%%%%%%%%%%%%%%%%%%%%%%%%%%%%%%%%%%%%%%
\Section{Getting and unpacking the sources}

You can download the sources from the same places as {\GAP}. So the main
FTP servers are:

\begintt
ftp://ftp-gap.dcs.st-and.ac.uk/pub/gap/gap4/
ftp://ftp.math.rwth-aachen.de/pub/gap4/
ftp://ftp.ccs.neu.edu/pub/mirrors/ftp-gap.dcs.st-and.ac.uk/pub/gap/gap4/
ftp://pell.anu.edu.au/pub/algebra/gap4/
\endtt

You need only one file with the name ``unipot1r1.zoo'' which is in the
subdirectory for the share packages. When you installed {\GAP} you used
the utility <unzoo> to unpack the distribution. You will need this here
again. See the {\GAP}-manual, chapter "ref:Installing GAP" for 
instructions on how to get and compile this. You now change your current 
directory to the ``pkg'' subdirectory of the location where you installed
{\GAP} (you typed an <unzoo>-command, then a new directory called ``gap4''
or something like that was created, this directory contains the ``pkg''
subdirectory). The standard location would be:
(do not type the prompt character `\#')
\begintt
# cd /usr/local/lib/gap4/pkg
\endtt

Now you extract the sources for the `unipot' share package:
\begintt
# unzoo -x unipot1r1.zoo
unipot/README     -- extracted as text
...
/bin/mkdir: cannot make directory `unipot': File exists
...
\endtt

Note that the warning is *not* serious.

The <unzoo> utility unpacks the files and stores them into the apropriate
subdirectories. `unipot' resides completely in the following subdirectory
(assuming standard location):

\begintt
/usr/local/lib/gap4/pkg/unipot
\endtt


%%%%%%%%%%%%%%%%%%%%%%%%%%%%%%%%%%%%%%%%%%%%%%%%%%%%%%%%%%%%%%%%%%%%%%%%%
\Section{Installing in a different than the standard location}

It could happen that you do not want to install `unipot' in its
standard location, perhaps because you do not want to bother your
system administrator and have no access to the {\GAP} directory. In
this case you can unpack `unipot' in any other location within a
``pkg'' directory with the <unzoo> command as described above. Let us
call this directory ``pkg'' for the moment. You get an ``unipot''
subdirectory with all the files of `unipot' in it. Then you follow the
standard procedure with following exceptions:

Say, the directory containing the ``pkg'' directory is
``/home/user/mygap''.
Note that you have either to edit the startup script ``gap.sh'':
 Add ``/home/user/mygap'' separating it with semicolon `;' from previous
 directories for the variable ``GAP_DIR''.
Or you have to start {\GAP} with following command line option:

\begintt
# gap4 -l "/usr/local/lib/gap4;/home/user/mygap"
\endtt


%%%%%%%%%%%%%%%%%%%%%%%%%%%%%%%%%%%%%%%%%%%%%%%%%%%%%%%%%%%%%%%%%%%%%%%%%
\Section{Loading unipot in GAP}

Add a line to the ``ALLPKG'' file in the ``pkg'' directory
\begintt
# cd /name-of-gap-directory/pkg
# echo unipot >> ALLPKG
\endtt

This makes the documentation of the package available in any {\GAP4}
session, even if the package is not loaded.
Like any other share package, `unipot' is loaded in {\GAP} with

\beginexample
gap> RequirePackage("unipot");
\endexample

within the {\GAP4} session.

If you have problems with this package, wish to make comments
or suggestions, or if you find bugs, please send e-mail to me.

Sergei Haller, mailto:`Sergei.Haller@math.uni-giessen.de'

Also, I would like to hear about applications of this package.
(See Preface.3)


%%%%%%%%%%%%%%%%%%%%%%%%%%%%%%%%%%%%%%%%%%%%%%%%%%%%%%%%%%%%%%%%%%%%%%%%%
%%
%E  install.tex . . . . . . . . . . . . . . . . . . . . . . . . ends here



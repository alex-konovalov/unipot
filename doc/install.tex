%%%%%%%%%%%%%%%%%%%%%%%%%%%%%%%%%%%%%%%%%%%%%%%%%%%%%%%%%%%%%%%%%%%%%%%%%
%%
%W  install.tex            UNIPOT documentation             Sergei Haller
%%
%%

%%%%%%%%%%%%%%%%%%%%%%%%%%%%%%%%%%%%%%%%%%%%%%%%%%%%%%%%%%%%%%%%%%%%%%%%%
\Chapter{Installing and Loading Unipot}

This document describes the procedure of installing the package.

Installing {\Unipot} should  be  easy  once  you  have  installed  {\GAP}
itself. We assume here that you want to install {\Unipot} in its standard
location,  which  is  in  the  `pkg'  subdirectory  of the  main  {\GAP4}
installation.


%%%%%%%%%%%%%%%%%%%%%%%%%%%%%%%%%%%%%%%%%%%%%%%%%%%%%%%%%%%%%%%%%%%%%%%%%
\Section{Overview}\null

You have to perform the following steps to install {\Unipot}:

\beginlist%unordered
\item{--} Get the sources.
\item{--} Unpack the sources with the `unzoo' utility.
\item{--} Optionally edit the `ALLPKG' file so that the {\Unipot}
     documentation will be available when {\GAP} starts up.
\endlist                                                                                                      


%%%%%%%%%%%%%%%%%%%%%%%%%%%%%%%%%%%%%%%%%%%%%%%%%%%%%%%%%%%%%%%%%%%%%%%%%
\Section{Getting and unpacking the sources}

You can download the sources  from the same places as {\GAP}. So the main
FTP servers are:

\beginlist%unordered
\item{$\bullet$}\URL{ftp://ftp-gap.dcs.st-and.ac.uk/pub/gap/gap4/}
\item{$\bullet$}\URL{ftp://ftp.math.rwth-aachen.de/pub/gap4/}
\item{$\bullet$}\URL{ftp://ftp.ccs.neu.edu/pub/mirrors/ftp-gap.dcs.st-and.ac.uk/pub/gap/gap4/}
\item{$\bullet$}\URL{ftp://web-serv.zsu.zp.ua/Public/Gap4/}
\endlist

You  need  only one  file with the name `unipot1r1.zoo'  which  is in the
subdirectory for the packages.  When  you  installed  {\GAP} you used the
utility  `unzoo'  to unpack the  distribution.  You  will  need this here
again.  See  {\GAP}  Reference manual, chapter "ref:Installing  GAP"  for
instructions on how to  get and compile it.  Go to the subdirectory `pkg'
of the main {\GAP} directory (when you installed the {\GAP} distribution,
a new  directory  called  `gap4r<X>'  for  some  <X>  was  created;  this
directory contains the `pkg'  subdirectory). The standard  location would
be: (do not type the prompt character `\#')

\begintt
# cd /usr/local/lib/gap4rX/pkg
\endtt

Then extract the sources of the {\Unipot} package:
\begintt
# unzoo -x unipot1r1.zoo
unipot/README     -- extracted as text
...
/bin/mkdir: cannot make directory `unipot': File exists
...
\endtt

Note that the warning is *not* serious.

The `unzoo' utility unpacks the files and stores them into the apropriate
subdirectories. {\Unipot} resides completely in the following subdirectory
(assuming standard location):

\begintt
/usr/local/lib/gap4rX/pkg/unipot
\endtt


%%%%%%%%%%%%%%%%%%%%%%%%%%%%%%%%%%%%%%%%%%%%%%%%%%%%%%%%%%%%%%%%%%%%%%%%%
\Section{Installing in a different than the standard location}

It could happen that you do not want to install {\Unipot} in its standard
location,  perhaps  because  you  do  not  want  to  bother  your  system
administrator and have  no access to the {\GAP} directory.  In  this case
just unpack  {\Unipot} in any other location within a `pkg'  directory as
described  above. E.g.  the directory containing  the  `pkg' directory is
`/home/user/mygap',       then       {\Unipot}       resides       inside
`/home/user/mygap/pkg/unipot'.

Note that you *either* have to edit the startup script `gap.sh':
\beginlist%unordered
   Add `/home/user/mygap' separating it with a semicolon (``;'')
   from previous directories for the variable `GAP_DIR'
\endlist
*or* you have to start {\GAP} with following command line option:
\begintt
# gap4 -l "/usr/local/lib/gap4rX;/home/user/mygap"
\endtt


%%%%%%%%%%%%%%%%%%%%%%%%%%%%%%%%%%%%%%%%%%%%%%%%%%%%%%%%%%%%%%%%%%%%%%%%%
\Section{Loading Unipot in GAP}

Add a line to the `ALLPKG' file in the `pkg' directory
\begintt
# cd /name-of-gap-directory/pkg
# echo unipot >> ALLPKG
\endtt

This makes the  documentation  of the package  available  in any  {\GAP4}
session,  even  if the  package  is not loaded. Like  any other  package,
{\Unipot} is loaded in {\GAP} with

\beginexample
gap> LoadPackage("unipot");
\endexample

within the {\GAP4} session.

If you have problems with this package, wish to make comments
or suggestions, or if you find bugs, please report them via

\URL{https://github.com/gap-packages/unipot/issues}

Also, I would like to hear about applications of this package.
(See also "Citing Unipot".)

%%%%%%%%%%%%%%%%%%%%%%%%%%%%%%%%%%%%%%%%%%%%%%%%%%%%%%%%%%%%%%%%%%%%%%%%%
%%
%W  preface.tex            UNIPOT documentation             Sergei Haller
%%
%H  $Id: preface.tex,v 2.1 2000/07/13 11:38:37 gc1007 Exp $
%%
%%

%%%%%%%%%%%%%%%%%%%%%%%%%%%%%%%%%%%%%%%%%%%%%%%%%%%%%%%%%%%%%%%%%%%%%%%%%
\Chapter{Preface}

`Unipot' is a share package for {\GAP4} \cite{GAP4}. This package is the
content of my diploma thesis \cite{SH2000}.

Let $U$ be a unipotent subgroup of a Chevalley group of Type $L(K)$. Then
it is generated by the elements $x_r(t)$ for all $r\in \Phi^+,t\in K$. The
roots of the underlying root system $\Phi$ are ordered according to the
height function. Each element of $U$ is a product of the root elements
$x_r(t)$. By the Theorem 5.3.3 from \cite{Carter72} each element of $U$
can be uniquely written as a product of root elements with roots in
increasing order. This unique form is called the canonical form.

The main purpose of this package is to compute the canonical form of an
element of the group $U$. For we have implemented the unipotent subgroups
of Chevalley groups and their elements as {\GAP} objects and installed
some operations for them. One method for the operation `Comm' uses the
Chevalley's commutator formula, which we have implemented, too.

\>Root Systems

We are using the root systems and the structure constants available in
{\GAP} via the simple Lie algebras. We also are using the ordering of
roots available in {\GAP}.

Note that the structure constants in {\GAP4.1} are not generated
corresponding to a Chevalley basis, so computations in the groups of
type $B_l$ may produce an error and computations in groups of types $B_l, 
C_l$ and $F_4$ may lead to wrong results. In the groups of other types
we haven't seen any wrong results but can not guarantee that all results
are correct.

In the revision 4.2 of {\GAP} the structure constants are generated
corresponding to a Chevalley basis, so that they meet all our assumtions.

Therefore the share package requires the revision 4.2 of {\GAP}.

%%%%%%%%%%%%%%%%%%%%%%%%%%%%%%%%%%%%%%%%%%%%%%%%%%%%%%%%%%%%%%%%%%%%
\> Future of `unipot'

In one of the future versions of the share package `unipot' we plan to
implement some other features. Here is a small list of them:
\beginlist
\item{--} {\GAP4.2} provides special root system objects. We should use
          them.
\item{--} Provide some root systems in common notations (like Carter or
          Bourbaki).
\item{--} Allow the user to provide his own table of structure constants.
\item{--} Provide whole Chevalley groups as {\GAP} objects
\item{--} Provide root subgroups
\item{--} The elements of Chevalley groups should act on the underlying
          simple Lie algebra as automorphisms
\item{--} There are many known properties of the Chevalley groups and
          their unipotent subgroups like simplicity, central series, etc.
          Implement them.
\endlist                                                                                                      


%%%%%%%%%%%%%%%%%%%%%%%%%%%%%%%%%%%%%%%%%%%%%%%%%%%%%%%%%%%%%%%%%%%%%%%%%

\>Citation

If you use `unipot' to solve a problem or publish some result that was
partly obtained using `unipot', I would appreciate it if you would cite
`unipot', just as you would cite another paper that you used. (Below is a
sample citation.)
Again I would appreciate if you could inform me about such a paper. 

Specifically, please refer to:
 
\begintt
[Hal00] Sergei Haller. Unipot --- a system for computing with elements
        of unipotent subgroups of Chevalley groups, Version 1.1.
        Justus-Liebig Universitaet Giessen, Germany, July 2000. 
        (ftp://ftp-pclabor.hrz.uni-giessen.de/SHadow/unipot/)
\endtt
 
(Should the reference style require full addresses please use:
``Arbeitsgruppe Algebra,
Mathematisches Institut,
Justus-Liebig Universit{\accent127a}t Gie{\ss}en,
Arndtstr. 2, 35392 Gie{\ss}en, Germany'')


%%%%%%%%%%%%%%%%%%%%%%%%%%%%%%%%%%%%%%%%%%%%%%%%%%%%%%%%%%%%%%%%%%%%%%%%%
%%
%E  preface.tex . . . . . . . . . . . . . . . . . . . . . . . . ends here


